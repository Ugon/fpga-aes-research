\section{System project}
\label{sec:system-project}
The research of this thesis is focused on evaluation and efficiency comparison of different approaches to implementing AES encryption. This chapter describes proposed AES circuits.

Proposed designs are:
\begin{enumerate}[noitemsep]
\item Simple asynchronous architecture
\item Pipelined unrolled architecture with stages of equal critical path
\item Pipelined unrolled architecture with stages containing one ALM
\item Pipelined rolled up architecture with stages containing one ALM
\item Pipelined rolled up architecture using on-chip memory
\end{enumerate}

\subsection{Simple asynchronous architecture}
The most basic approach to implementing AES encryption module is to utilize only combinatorial logic without any optimizations. Such approach was used in Author's BsC thesis \cite{inzynierka} and provides a good baseline for this research.

Asynchronous AES encryption circuit is not clocked and does not use pipelining. Encryption is done at the speed of signal propagation. As Author showed in his BsC thesis \cite{inzynierka}, this design is capable of encrypting \textbf{20M blocks per second}. Compilation report shows that it uses \textbf{27462} (excluding 806 elements required for testing circuit - section \ref{sec:testing-methodology}) FPGA ALMs.

This design is described in detail in Author's past research - his BSc thesis \cite{inzynierka}.

% To test this design for maximum speed of operation it was placed into testing framework described in Chapter \ref{sec:testing-methodology} and clocked. Because this AES circuit is asynchronous, entire enryption algorithm was executen in a single clock cycle of the testing framework. 

% Testing simple asynchronous AES encryption module showed that it is capable of encryptiong only 20M blocks per second and uses 123421????????????? FPGA ALMs. 
% Analysis of compilation reports showed that when increasing frequency, in order to meet timing constraints logic duplication was performed during compilation.

% The conclusion of analysing this approach is that implementing AES encryption in FPGA as an asynchronous circuit is possible, but there is room for optimizations.

\subsection{Pipelined unrolled architecture with stages of equal critical path length}
To achieve high throughput in FPGA designs pipelining is often used \cite[Chapter 1]{kilts2007advanced}. AES encryption algorithm consists of 14 rounds, 13 of which are exactly the same, which means that it can be implemented in two ways:
\begin{description}
\item[Unrolled pipeline] -- there are no loops in the design, parts of the design are not reused. This approach is good for achieving high throughput at the cost of high usage of FPGA resources.
\item[Rolled up pipeline] -- the design contains loops, AES round implementation is reused. This approach is not capable of achieving throughput as high as unrolled pipeline, but its usage of FPGA resources is considerably lower.
\end{description}

This design proposal is based on utilising an unrolled pipeline. Authors of \cite{vlsi} proposed such an architecture divided into pipelined stages of equal critical path length. The fastest of the designs they proposed \cite[Fig. 11]{vlsi} used $r=7$ pipeline stages in each round. This design will be implemented as an entry point to this research of throughput-optimized designs as proposed in \cite{vlsi}.


\subsection{Pipelined unrolled architecture with stages containing one ALM}
Next AES circuit proposal is an attempt to improve the design proposed in \cite{vlsi}. Author suspects that partitioning design into pipelining stages based on critical path length may not be an optimal solution for FPGA platform, because of FPGA's low level properties - utilisation of look-up tables for implementing combinatorial logic in particular.

This design proposal is a modification of the one described in \cite{vlsi}. AES rounds will be partitioned into pipeline stages so that combinatorial logic in each stage can be fit into only one FPGA LUT. The reasoning behind this approach is that higher frequencies could be achieved by avoiding pipeline stages which consist of more than one FPGA LUT, which introduces extra delay caused by additional routing.

FPGA technology details relevant to this approach will be discussed in detail in section \ref{sec:low-level-fpga} and implementation details in \ref{sec:implementation-pipe-unrolled-one-alm}. 


\subsection{Pipelined looped architecture with stages containing one ALM}

speed: 467MHz\\
LE: 2234


\subsection{Pipelined rolled up architecture using on-chip memory}
TBD
